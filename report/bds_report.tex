\documentclass[conference]{IEEEtran}
\IEEEoverridecommandlockouts
% The preceding line is only needed to identify funding in the first footnote. If that is unneeded, please comment it out.
\include{../solutions}
\usepackage{cite}
\usepackage{amsmath,amssymb,amsfonts}
% \usepackage{algorithmic}
\usepackage{graphicx}
\usepackage{textcomp}
\usepackage{xcolor}
\usepackage{tabularx}
% \usepackage{hyperref}

\def\BibTeX{{\rm B\kern-.05em{\sc i\kern-.025em b}\kern-.08em
    T\kern-.1667em\lower.7ex\hbox{E}\kern-.125emX}}

\title{\LARGE{Radiogenomic Prediction of Breast Cancer Subtypes \\ Using the TCGA Dataset}}

\author{
    \begin{tabular}{c}
        \begin{tabular}{cc}
            \begin{tabular}{c} 
                \IEEEauthorblockN{Lucas Fayolle} \\
                \vspace{-8mm}
                \IEEEauthorblockA{\textit{lfayoll@etsinf.upv.es}}
            \end{tabular} &
            \begin{tabular}{c} 
                \vspace{-4.5mm}
                \IEEEauthorblockN{Jose Valero Sanchis} \\
                \vspace{-8mm}
                \IEEEauthorblockA{\textit{jvalsan@etsinf.upv.es}}
            \end{tabular}
        \end{tabular}
    \end{tabular}
}

\begin{document}

\maketitle

\begin{abstract}
Lorem ipsum dolor sit amet, consectetur adipiscing elit. Sed porta, ante at finibus egestas, neque neque lacinia ipsum, nec accumsan turpis mauris in est. Sed mollis consectetur felis, id fringilla lacus efficitur id. Maecenas iaculis mattis lacus, eget faucibus est maximus in. Pellentesque pulvinar tortor neque, in rhoncus lorem consequat ac. Ut varius urna et lorem laoreet, lacinia venenatis lacus elementum. In vel ante eget diam facilisis consequat vitae eu mauris. 
\end{abstract}

\begin{IEEEkeywords}
Mathematical Optimization, ...
\end{IEEEkeywords}

%%%%%%%%%%%%%%%%%%%%%%%%%%%%%%%%%%%%%%%%%%%%%%%%%%%%%%%%%%%%%%%%%%%%%%%%%%%%%%%
%                                                                                          INTRODUCTION                                                                                                                                         %
%%%%%%%%%%%%%%%%%%%%%%%%%%%%%%%%%%%%%%%%%%%%%%%%%%%%%%%%%%%%%%%%%%%%%%%%%%%%%%%

\section{Introduction}


% FALTA DEFINIR ESTRUCTURAS



%%%%%%%%%%%%%%%%%%%%%%%%%%%%%%%%%%%%%%%%%%%%%%%%%%%%%%%%%%%%%%%%%%%%%%%%%%%%%%%
%                                                                                                               CONCLUSION                                                                                                                         %
%%%%%%%%%%%%%%%%%%%%%%%%%%%%%%%%%%%%%%%%%%%%%%%%%%%%%%%%%%%%%%%%%%%%%%%%%%%%%%%
 
\section{Conclusions}



%%%%%%%%%%%%%%%%%%%%%%%%%%%%%%%%%%%%%%%%%%%%%%%%%%%%%%%%%%%%%%%%%%%%%%%%%%%%%%%
%                                                                                                        REFERENCES                                                                                                                                  %
%%%%%%%%%%%%%%%%%%%%%%%%%%%%%%%%%%%%%%%%%%%%%%%%%%%%%%%%%%%%%%%%%%%%%%%%%%%%%%%
\begin{thebibliography}{00}
\bibitem{b1} S. I. Gass, ``On the division of police districts into patrol beats,'' in *Proceedings of the 1968 23rd ACM National Conference*, New York, NY, USA: Association for Computing Machinery, 1968, pp. 459--473. [Online]. Available: https://doi.org/10.1145/800186.810609
\end{thebibliography}

\end{document}