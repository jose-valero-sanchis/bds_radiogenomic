\section{Conclusions}

%------------------------------------------------

\begin{frame}
    \frametitle{Key findings}
    \vspace{3mm}

    \begin{itemize}
        \item Combining radiomic, clinical, and genomic data enhances the prediction of molecular subtypes of breast cancer.
	\vspace{4mm}
        \item The \textbf{Radiomic + Full Clinical Data} model achieved the highest performance, with an F1-score of 0.78, highlighting the importance of hormone receptor status and other clinical variables in subtype classification.
	\vspace{4mm}
        \item The inclusion of multigenic assays improved the model’s performance, with the \textbf{Radiomic + Multigenic model} achieving an F1-score of 0.74, demonstrating the complementarity between genomic data and radiomic features.
    \end{itemize}

    \vfill 
\end{frame}

%------------------------------------------------

\begin{frame}
    \frametitle{Study limitations and suggestions for future work}
    \vspace{3mm}

        \textbf{Study Limitations}
 	\vspace{2mm}
        \begin{itemize}
            \item Small dataset size.
            \item Limited computational resources.
        \end{itemize}

 	\vspace{5mm}

        \textbf{Suggestions for Future Work}
 	\vspace{2mm}
        \begin{itemize}
            \item Increased data collection.
            \item Better data organization and integration.
            \item Extraction of radiomic features from raw images.
            \item Hyperparameter tuning.
        \end{itemize}

    \vfill 
\end{frame}